
	At the UCSB Media Arts and Technology program, efforts between a multitude of architects, professors, graduate students have aggregated into an innovative new medium of visualizing complex multi-dimensional data. The program features a 3-story cube housing an aluminum sphere lined with 26 3D projectors, sound-absorption material, and an array of speakers designed to produce an immersive environment for scientists. A team of computer music researchers generates the audio for the sphere--providing auditory constructs that visualize data from neural networks to electron spins. This use of musical representations is the focus of this annotated bibliography.

The rest of this paper is organized as follows.  Section~\ref{sec:Background} explains the background for the MAT program and how its contributions have aided researchers and scientists alike. Section~\ref{sec:ComputerMusic} touches on Gamma and the computer music research at UCSB. In Section ~\ref{sec:Allosphere}, we describe the methods in which computer music are implemented in the allosphere.

