Roberts, Charlie, and Wright M. "The Web Browser As Synthesizer And Interface."\cite{RobertsCharlieWright}

The usage of two new javascript libraries called Gibberish.js and Interface.js allow the web browser to act as a synthesizer. Usage of websockets allows for efficient communication over standard AJAX procedures in javascript. These new libraries promote an improvement to Google's Web Audio API-- more sample--accurate timing as well as feedback networks. These two allow far improved latency in browser-based music synthesis. Today, both of these libraries are open source and drastically add to the Web Audio API.

Roberts, C., Forbes, A. and Höllerer, T. "Enabling Multimodal Mobile Interfaces for Interactive Musical Performance."\cite{RobertsForbesHollerer}

A new mobile app called Control allows musical performances from a mobile interface. The application uses JSON to create complex structures capable of multimodal signals. MIDI signals can be sent wirelessly from a mobile device such as an Iphone aggregated from a variety of sensory inputs such as touch and accelerometer. 

